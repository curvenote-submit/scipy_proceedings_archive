\begin{lstlisting}[language=python]
@cuda.jit
def process_kernel_no_buffer(
    rng_states, N, n_neutrons_per_thread,
    args
):
    args0, args1, args2, offsets, rotmats, neutron_counter = args
    thread_index = cuda.grid(1)
    start_index = thread_index*n_neutrons_per_thread
    end_index = min(start_index+n_neutrons_per_thread, N)
    neutron = cuda.local.array(shape=10, dtype=NB_FLOAT)
    r = cuda.local.array(3, dtype=NB_FLOAT)
    v = cuda.local.array(3, dtype=NB_FLOAT)
    for neutron_index in range(start_index, end_index):
        cuda.atomic.add(neutron_counter, 0, 1)
        propagate0(thread_index, rng_states,  neutron, *args0)
        applyTransformation(neutron[:3], neutron[3:6],
                            rotmats[0], offsets[0], r, v)
        propagate1(neutron, *args1)
        applyTransformation(neutron[:3], neutron[3:6],
                            rotmats[1], offsets[1], r, v)
        propagate2(neutron, *args2)

from mcvine.acc.components.sources.SourceBase import SourceBase
class _Base(SourceBase): # has to be named Base in definition
    def __init__(self, instrument):
        offsets, rotmats = calcTransformations(instrument)
        self.neutron_counter = neutron_counter = np.zeros(1, dtype=int)
        self.propagate_params = tuple(
            c.propagate_params for c in instrument.components)
        self.propagate_params += (offsets, rotmats, neutron_counter)
        return
    def propagate(self): return
InstrumentWrapper = _Base
InstrumentWrapper.process_kernel_no_buffer = process_kernel_no_buffer

def run(ncount, ntotalthreads=None, threads_per_block=None, **kwds):
    instrument = loadInstrument(script, **kwds)
    iw = InstrumentWrapper(instrument)
    iw.process_no_buffer(ncount, ntotalthreads=ntotalthreads,
                         threads_per_block=threads_per_block)
    processed = iw.neutron_counter[0]
    saveMonitorOutputs(instrument, scale_factor=1.0/ncount)
\end{lstlisting}
